\documentclass[12pt,a4paper]{article}

\usepackage[utf8]{inputenc}
\usepackage[T1]{fontenc}
\usepackage[margin=1in]{geometry}
\usepackage{xcolor}
\usepackage{hyperref}
\usepackage{enumitem}
\usepackage{titlesec}
\usepackage{fancyhdr}
\usepackage{graphicx}
\usepackage{tcolorbox}
\usepackage{listings}

% --- DESIGN ---
\definecolor{primary}{RGB}{30, 144, 255}
\definecolor{glass}{RGB}{240, 240, 240}

\hypersetup{
    colorlinks=true,
    linkcolor=primary,
    urlcolor=primary,
}

\pagestyle{fancy}
\fancyhf{}
\rhead{\textit{STARtTalking Proposed System}}
\lhead{\textbf{v1.0}}
\rfoot{\thepage}

\titleformat{\section}{\Large\bfseries\color{primary}}{}{0.5em}{}
\titleformat{\subsection}{\large\bfseries\color{darkgray}}{}{0.5em}{}

\title{
    \Huge \textbf{PROPOSED SYSTEM DOCUMENT} \\
    \Large \textit{STARtTalking: AI-Powered Soft-Skills Application}
}
\author{\textbf{Master Architect / Glue Engineer}}
\date{\today}

\begin{document}

\maketitle
\newpage
\tableofcontents
\newpage

\section{Introduction}
The STARtTalking platform is a high-fidelity, cloud-native soft-skills development application designed to address the growing demand for effective communication in professional environments. In an era where technical skills are often secondary to the ability to communicate ideas effectively, STARtTalking provides a personalized, AI-driven environment for mastering the art of storytelling and presentation.

\subsection{Background and Justification}
Traditional soft-skills training methods, such as in-person workshops or peer-to-peer coaching, are often difficult to scale, expensive, and lack objective, data-driven feedback. Digital learning platforms (MOOCs) provide theoretical knowledge but fail to offer a safe space for "active practice." The proposed system leverages cutting-edge web technologies (React 19, Vite 7) and Artificial Intelligence (LLMs and Speech-to-Text) to provide an immersive, real-time training experience that is accessible globally.

\subsection{Goal of the Proposed System}
The primary objective is to replace traditional, static presentation tools with an interactive feedback loop. By utilizing Speech-to-Text (STT) and Natural Language Processing (NLP), the system aims to quantify subjective soft-skills such as clarity, empathy, pacing, and tone. It specifically adopts the \textbf{STAR Method} (Situation, Task, Action, Result) as the foundational framework for story construction.

\section{System Scope and Objectives}
The scope of the STARtTalking system encompasses the entire value chain of soft-skill practice, starting from initial story conceptualization within the STAR framework, through real-time practice sessions, to final detailed AI-generated performance analytics.

\subsection{Functional Objectives}
\begin{enumerate}
    \item \textbf{Secure Authentication:} Implementation of a robust security layer with JWT-based sessions and vertical separation of user data.
    \item \textbf{STAR Story Engineering:} A guided builder that decomposes stories into Situation, Task, Action, and Result components.
    \item \textbf{Vertical Module Separation:} Six distinct functional modules (Login, Register, Dashboard, Story-Builder, Practice, Feedback) ensuring high maintainability and developer isolation.
    \item \textbf{Immersive Practice Environment:} Low-latency voice capture using the Web Audio API with visual feedback (waveforms and pulse indicators).
    \item \textbf{AI-Driven Feedback Engine:} Integration with cloud-scale AI (OpenAI/Azure) to perform sentiment analysis, keyword extraction, and clarity scoring.
\end{enumerate}

\subsection{Non-Functional Objectives}
\begin{itemize}
    \item \textbf{UI Aesthetics (Liquid Glass):} Implementation of a premium "iPhone Flagship" design language, utilizing high-quality blurs, squircle corners, and organic transitions.
    \item \textbf{Performance:} Initial page load under 1.5 seconds and real-time UI animation at 60fps.
    \item \textbf{Scalability:} A micro-frontend-ready architecture allowing individual modules to be scaled or replaced independently.
    \item \textbf{Privacy:} Client-side audio buffering to ensure data is only transmitted during active submission phases.
\end{itemize}

\section{Requirements Analysis}

\subsection{Software Requirements}
\begin{itemize}
    \item \textbf{Operating System:} Cross-platform (Windows, macOS, Linux, iOS, Android).
    \item \textbf{Environment:} Node.js v18.0+, NPM v9.0+.
    \item \textbf{Frameworks:} React 19.0.0, Vite 7.3.1.
    \item \textbf{Styling:} Tailwind CSS v4 (with @tailwindcss/vite).
    \item \textbf{APIs:} Web Audio API, MediaDevices API, Browser Fetch API.
\end{itemize}

\subsection{Hardware Requirements}
\begin{itemize}
    \item \textbf{Processor:} Dual-core 2.4GHz or equivalent (ARM or x86).
    \item \textbf{RAM:} 4GB Minimum (8GB Recommended for seamless AI visualization).
    \item \textbf{Peripherals:} Built-in or external high-quality microphone.
    \item \textbf{Display:} Minimum 1280x720 resolution (Optimized for Mobile/Touch).
\end{itemize}

\section{System Architecture}
The proposed architecture is based on a decoupled client-server model, emphasizing a "Glue Engineer" methodology where the core system architecture is centralized but modules are developed independently.

\subsection{Frontend Architecture: Vertical Isolation}
As the Lead Architect, I have established a 6-module folder architecture. Each module operates in vertical isolation, meaning that a developer working on the `Practice` module cannot modify the `Dashboard` logic. This ensures that the system architecture remains intact during the parallel development phase.

\begin{tcolorbox}[colback=primary!5, colframe=primary, title=Architecture Placeholder]
    \centering
    \textit{[INSERT MASTER SYSTEM ARCHITECTURE DIAGRAM HERE]} \\
    \textit{Refer to PlantUML code in DIAGRAM\_CODES.md}
\end{tcolorbox}

\section{Module-Wise Specification and Development Guardrails}

\subsection{Module 1: Authentication (Login/Register)}
Handles user sessions through the "Liquid Glass" entry point. It utilizes secure password transmission and provides clear visual feedback for authentication states.

\subsection{Module 2: Dashboard (The Command Center)}
The user dashboard centralizes analytics from all practice sessions. It utilizes custom-built widgets (Stats Cards) that follow the design tokens documented in `DesignTokens.md`.

\subsection{Module 3: STAR Story Builder}
The "Brain" of the pre-practice phase. It forces the user to structure their professional experiences using the STAR method, which is the industry standard for competency-based interviews.

\subsection{Module 4: Training Zone (Practice)}
The immersive practice module. It utilizes CSS-based pulse animations and real-time status indicators (Recording, Processing) to keep the user engaged.

\subsection{Module 5: Global Feedback Engine}
The integration layer that compiles data from the AI analysis and presents it in a multi-category scorecard (Clarity, Tone, Empathy).

\section{Proposed Development Timeline}
The project follows a two-phase timeline:
\begin{description}
    \item[Day 1-2: Architect Phase] Establishment of the skeleton, routing, and design tokens (Completed).
    \item[Day 3-10: Module Phase] Parallel development of feature logic by the engineering team (In Progress).
    \item[Day 11-14: Glue Phase] Integration of modules, final flow refinement, and deployment (Planned).
\end{description}

\section{Design Language: Liquid Glass}
The "Liquid Glass" UI is the hallmark of the proposed system. This aesthetic utilizes depth, translucency (blur), and organic shapes to provide a flagship mobile experience on a web platform.

\begin{lstlisting}[language=JSX, caption=Liquid Glass Token Implementation]
.glass-panel {
    background: rgba(255, 255, 255, 0.7);
    backdrop-filter: blur(20px);
    border-radius: 2.5rem;
}
\end{lstlisting}

\section{Conclusion and Future Roadmap}
The proposed STARtTalking system is more than a simple web application; it is an intelligent coaching ecosystem. Future iterations will include multi-user collaboration and video-based non-verbal analysis.

\begin{tcolorbox}[colback=primary!5, colframe=primary, title=UML Placeholder]
    \centering
    \textit{[INSERT USE CASE AND CLASS DIAGRAMS HERE]} \\
    \textit{Refer to PlantUML sections 2 and 4 in DIAGRAM\_CODES.md}
\end{tcolorbox}

\end{document}
