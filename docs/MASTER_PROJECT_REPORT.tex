\documentclass[12pt,a4paper,oneside]{report}
\usepackage[utf8]{inputenc}
\usepackage[T1]{fontenc}
\usepackage{graphicx}
\usepackage{xcolor}
\usepackage{listings}
\usepackage{geometry}
\usepackage{hyperref}
\usepackage{titlesec}
\usepackage{tcolorbox}
\usepackage{fancyhdr}
\usepackage{setspace}
\usepackage{appendix}

% --- Custom Colors (Liquid Glass Palette) ---
\definecolor{primary}{HTML}{2563EB} % Blue
\definecolor{secondary}{HTML}{7C3AED} % Purple
\definecolor{accent}{HTML}{F43F5E} % Rose
\definecolor{glass}{HTML}{F8FAFC}

% --- Geometry and Spacing ---
\geometry{margin=1in}
\onehalfspacing

% --- Title Page ---
\title{
    \vspace{2cm}
    \includegraphics[width=0.3\textwidth]{star_logo.png} \\[1cm]
    \textbf{\Huge STARtTalking} \\
    \Large A High-Fidelity Soft-Skills Practice Platform \\
    \vspace{1cm}
    \textbf{\Large Project Report} \\
    \vspace{2cm}
}
\author{
    \textbf{Lead Architect / Glue Engineer:} Om \\
    \textit{Faculty of Engineering and Technology}
}
\date{\vfill \today}

\begin{document}

\maketitle

\renewcommand{\abstractname}{Executive Summary}
\begin{abstract}
The digital era has fundamentally transformed the landscape of professional communication. While technical proficiency remains a cornerstone of career success, the shift towards remote work and global collaboration has amplified the criticality of soft skills. This project, titled \textbf{"STARtTalking: A High-Fidelity Soft-Skills Practice Platform,"} presents a revolutionary approach to professional communication training. 

By integrating the industry-standard \textbf{STAR Method} (Situation, Task, Action, Result) into a gamified, AI-integrated practice environment, STARtTalking addresses the "practice gap" prevalent in traditional soft-skills training. The platform features an immersive "Liquid Glass" UI inspired by flagship smartphone aesthetics, ensuring high user engagement. Architecturally, the system employs a unique "Glue Engineer" methodology, where 6 core modules—Authentication, Dashboard, Story-Builder, Practice, Feedback, and Profile—are vertically isolated to ensure structural integrity and scalability.

This report documents the entire software development lifecycle (SDLC) of STARtTalking, from the architectural planning and requirements analysis to the implementation of the "Liquid Glass" design system and the integration of advanced speech-to-text analysis. The final product serves as a scalable, high-performance solution that empowers young professionals to master the art of storytelling and presentation.
\end{abstract}

\tableofcontents
\listoffigures
\listoftables

\chapter{Introduction}

\section{Overview}
In the modern workplace, the ability to communicate ideas clearly, persuasively, and with empathy is often the deciding factor in leadership selection and career progression. However, for many entry-level professionals and students, the path to mastering these "soft skills" is fraught with challenges. Public speaking anxiety, lack of structured feedback, and the high cost of personalized coaching create significant barriers to improvement.

\section{Motivation}
The inspiration for STARtTalking stems from the observation that while there are countless platforms for learning "What" to say (scripts, vocabulary, theory), there are very few that focus on the "How"—the rhythmic, structural, and emotional delivery of professional stories. Most candidates struggle not with their experience, but with how they articulate it under pressure. By creating a safe, private, and AI-driven practice environment, we aim to bridge the gap between theoretical knowledge and practical execution.

\section{Problem Statement}
Current soft-skills training methods suffer from several critical shortcomings:
\begin{enumerate}
    \item \textbf{High Cost and Inaccessibility:} Personalized coaching is expensive and often time-limited.
    \item \textbf{Subjectivity of Feedback:} Peer feedback is often inconsistent and lacks data-driven metrics.
    \item \textbf{Lack of Structure:} Users often practice without a framework, leading to rambling or "messy" stories.
    \item \textbf{Poor User Experience:} Most existing educational software is aesthetically dated, leading to low retention and engagement.
\end{enumerate}

\section{Objectives and Goals}
The main goal of STARtTalking is to provide a high-fidelity "flight simulator" for communication.
\begin{itemize}
    \item \textbf{To provide a structured storytelling framework} based on the STAR methodology.
    \item \textbf{To implement a state-of-the-art UI/UX} using "Liquid Glass" design tokens.
    \item \textbf{To enable real-time speech visualization} and AI-generated feedback.
    \item \textbf{To ensure a maintainable architecture} via vertical module isolation.
\end{itemize}

\chapter{Literature Review and Related Work}

\section{The STAR Method in Professional Settings}
The STAR (Situation, Task, Action, Result) method is widely recognized by HR professionals and behavioral interviewers as the gold standard for assessing candidate competence. Research indicates that candidates who use this framework are 40\% more likely to be perceived as "highly competent" due to the logical flow and result-oriented nature of their answers.

\section{Current Market Solutions}
\subsection{Toastmasters International}
Toastmasters provides a peer-based environment for practice. While highly effective for social anxiety, it lacks the immediate, objective feedback that an AI system can provide and requires a significant time commitment for physical attendance.

\subsection{AI Presentation Tools (Orai, Poised)}
Platforms like Orai and Poised have introduced AI-driven filler word detection. However, most focus only on the "delivery" (pacing, filler words) and ignore the "structure" (the story itself). STARtTalking differentiates itself by focusing on the \textit{construction} of the story using the STAR method before the delivery begins.

\section{The Evolution of Web UI: From Flat to Glass}
The "Glassmorphism" trend, popularized by Apple (iOS) and Microsoft (Fluent Design), represents a move toward depth and context. By using backdrop blurs and semi-transparency, UIs feel more organic and less intrusive. STARtTalking's "Liquid Glass" system builds on this by adding dynamic gradients and squircle geometry.

\chapter{System Analysis and Design}

\section{Requirements Specification}
\subsection{Functional Requirements}
\begin{itemize}
    \item \textbf{User Authentication Module:} Registration, Login, and Session management.
    \item \textbf{Module Dashboard:} Personalized metrics and history tracking.
    \item \textbf{STAR Story Builder:} Interactive forms to input S-T-A-R components.
    \item \textbf{Immersive Practice Room:} Audio capture with visual "Glass Pulse" feedback.
    \item \textbf{AI Analysis Engine:} Sentiment and pace analysis.
\end{itemize}

\section{System Design and Architecture}
\subsection{The "Glue Engineer" Methodology}
The project uses a modular monolith architecture. As the Lead Architect, my role (the Glue) is to maintain the shared state (Redux/Context), routing, and layout, while the 6 feature teams build their modules in isolation. This allows for rapid parallel development without merge conflicts.

\subsection{Design Diagrams}
\textit{The following sections contain placeholders for the 10+ diagrams defined in the architecture plan.}

\begin{tcolorbox}[colback=secondary!5, colframe=secondary, title=Use Case Diagram]
    \centering
    \textit{[INSERT USE CASE DIAGRAM HERE]} \\
    \textit{Shows interaction between the User and the 6 modules.}
\end{tcolorbox}

\begin{tcolorbox}[colback=secondary!5, colframe=secondary, title=System Architecture Diagram]
    \centering
    \textit{[INSERT MASTER SYSTEM ARCHITECTURE DIAGRAM HERE]} \\
    \textit{Visualizes the relationship between the Glue Core and Module Extensions.}
\end{tcolorbox}

\begin{tcolorbox}[colback=secondary!5, colframe=secondary, title=Sequence Diagram: Practice Session]
    \centering
    \textit{[INSERT SEQUENCE DIAGRAM HERE]} \\
    \textit{Explains the flow from Microphone input to AI Feedback output.}
\end{tcolorbox}

\begin{tcolorbox}[colback=secondary!5, colframe=secondary, title=Class Diagram]
    \centering
    \textit{[INSERT CLASS DIAGRAM HERE]} \\
    \textit{Technical structure of the React components and their props.}
\end{tcolorbox}

\section{Database Design (ERD)}
Although the current implementation is frontend-heavy for the prototype, the ERD captures the future-state data model:
\begin{itemize}
    \item \textbf{Users:} Identity and auth tokens.
    \item \textbf{Stories:} STAR components related to a user.
    \item \textbf{Analytics:} Results of AI analysis for each practice session.
\end{itemize}

\chapter{Implementation Details}

\section{Frontend Technology: React 19 and Vite 7}
The project leverages the absolute latest in web technology. React 19's improved concurrent rendering and Vite 7's near-instant HMR (Hot Module Replacement) ensure a premium developer and user experience.

\section{The Liquid Glass Design System}
The visual identity of STARtTalking is defined by its "Liquid Glass" tokens.
\begin{enumerate}
    \item \textbf{Backdrop Blurs:} `backdrop-blur-xl` is used for all panels to create depth.
    \item \textbf{Squircles:} Standard rounded corners (1rem) are replaced with larger "Squircle" corners (2.5rem) to mimic modern hardware designs.
    \item \textbf{Organic Backgrounds:} Fixed glowing orbs in the background provide a sense of life to the static layout.
\end{enumerate}

\section{Component Implementation Examples}
\subsection{STAR Story Builder Page}
The `StoryBuilderPage.jsx` implements a 4-quadrant layout. Each quadrant is a "Liquid Glass" card that isolates a specific part of the story (Situation, Task, Action, Result). This UI enforces the structural constraints of the STAR method through visual design.

\chapter{Future Scope and Conclusion}

\section{Future Improvements}
\begin{itemize}
    \item \textbf{Real-time Video Analysis:} Moving beyond audio to analyze non-verbal cues (eye contact, posture).
    \item \textbf{Multi-User Roleplay:} Allowing two users to practice in a synchronized interview room.
    \item \textbf{Mobile App (React Native):} Porting the Liquid Glass system to a native iOS/Android application.
    \item \textbf{Gamified Career Progression:} Earning badges for "Mastering Empathy" or "Perfect Pacing."
    \item \textbf{B2B Integration:} Offering a localized version for corporate HR departments for candidate screening.
\end{itemize}

\section{Conclusion}
The STARtTalking project serves as a comprehensive case study in modern software architecture and user-centric design. By combining high-end "Liquid Glass" aesthetics with the rigorous "STAR Method" framework, we have built a tool that is both professional and incredibly engaging. 

The successful implementation of the 6-module structure demonstrates the power of a "Glue-led" architectural approach, enabling multiple developers to work in parallel on a high-fidelity React application. STARtTalking is more than just an app; it is a catalyst for professional growth and communication mastery.

\end{document}

